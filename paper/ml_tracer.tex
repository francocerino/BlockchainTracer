\documentclass[a4paper,12pt]{article}
\usepackage[utf8]{inputenc}
\usepackage{amsmath}
\usepackage{graphicx}
\usepackage{hyperref}
\usepackage{listings}
\usepackage{color}
\usepackage{hyperref}
\usepackage{xcolor}

%\usepackage{makeidx}
%\usepackage{titleref}

\title{Blockchain para Mejorar la Reproducibilidad del Aprendizaje Automático}

%\makeindex

\date{}

%\author{Franco Cerino}
%\date{\today}


\begin{document}

\maketitle
\begin{abstract}

La reproducibilidad de un experimento es un principio fundamental en la ciencia, que da lugar a validar y garantizar la integridad de resultados obtenidos. En la actualidad, los modelos de Machine Learning (ML) son ampliamente utilizados para obtener resultados y generar avances en una gran diversidad de áreas, tanto de la ciencia, como de la industria. Por lo tanto, la reproducibilidad de los resultados obtenidos con estos modelos es de gran relevancia para permitir la credibilidad de los mismos. Actualmente existen desafíos de la reproducibilidad en ML, ya que para poder llevar a cabo esta tarea puede ser necesario tener informacion relacionada al modelo entrenado, como el hardware subyacente, versiones de librerias, inicializaciones con elementos aleatorios, optimizaciones realizadas y documentación. En este trabajo se propone a la tecnología blockchain como una solución para favorecer la reproducibilidad de estos modelos. La tecnología blockchain se caracteriza por ser una base de datos transparente, trazable, altamente disponible e inmutable. Por lo tanto, se plantea utilizarla para mejorar la integridad y disponibilidad de las características de los procesos llevados a cabo en un experimento de ML. Se presenta como soluciones a tecnologías Blockchain basadas en la Ethereum Virtual Machine (EVM) o a la Blockchain Algorand, que dispone de la Algorand Virtual Machine (AVM).
\end{abstract}
\newpage
\tableofcontents

\section{Introducción}

En los ámbitos de la ciencia y la industria, la reproducibilidad de experimentos permite aumentar la confianza y validar la consistencia de los resultados obtenidos, dando lugar a la posibilidad de generar avances y colaboraciones con mayor facilidad. 

\textcolor{blue}{[Al momento de escribir tener en cuenta diferencias entre replicabilidad y reproducibilidad:}

\href{https://www.ncbi.nlm.nih.gov/books/NBK547546/#:~:text=B2%3A%20%E2%80%9CReproducibility%E2%80%9D%20refers%20to,using%20the%20original%20author's%20artifacts.}
{\textcolor{blue}{Defining Reproducibility and Replicability}}

\href{https://cs.uwaterloo.ca/~brecht/courses/854-Experimental-Performance-Evaluation-2017/readings/replicability-is-not-reproducability.pdf}{\textcolor{blue}{Replicability is not Reproducibility:
Nor is it Good Science ]}}

En particular, en el área del aprendizaje automático, el progreso también depende de la reproducibilidad de los modelos y resultados con los que se ha tratado anteriormente. A pesar de la importancia de la reproducibilidad en ML, se enfrentan desafios debido a la complejidad que pueden presentar los entornos experimentales, como las dependencias de código y diferencias en versiones de librerías. Por otro lado, al momento de reproducir un modelo, los datos utilizados para entrenarlo pueden no estar disponibles o los preprocesamientos utilzados pueden no ser explicitados de forma suficiente. Además, a menudo los algoritmos de ML tienen componentes aleatorias que pueden influir en los resultados, por lo cual es relevante tener contabilidad de cómo estas son definidas.

También se resalta la importancia del código abierto y la disposición de los datos en forma pública para favorecer la reproducibilidad.

\textcolor{blue}{Expandir qué se habla en las referencias que están a continuación y cómo se alinean con la problemática mencionada.}

\begin{itemize}
    \item \href{https://www.jmlr.org/papers/volume22/20-303/20-303.pdf}{Improving Reproducibility in Machine Learning Research (2021)}
    \item \href{https://arxiv.org/pdf/2307.10320}{Reproducibility in Machine Learning-Driven
Research (2023)}

    \item \href{https://www.cell.com/patterns/pdf/S2666-3899(23)00159-9.pdf}{Leakage and the reproducibility crisis in machinelearning-based science (2023)}
    \item \href{https://arxiv.org/pdf/2308.07832}{reforms: Reporting Standards for Machine Learning Based Science (2023)}
    \item 
    \href{https://www.mdpi.com/2504-2289/5/2/20}{Traceability for Trustworthy AI: A Review of Models and Tools (2021)}
    \item \href{https://pytorch.org/docs/stable/notes/randomness.html}{Reproducibility in PyTorch}
    \item \href{https://content.iospress.com/download/informatica/infor553?id=informatica%2Finfor553}{Advancing Research Reproducibility in Machine
Learning through Blockchain Technology (2024)}
\end{itemize}


\section{Blockchain como herramienta para mejorar la reproducibilidad}

\textcolor{blue}{[Agregar sección con una introducción más técnica de Blockchain. 
De todas formas mejor comenzar con las ideas mas importantes a tener en cuenta y luego enfocarse en introducciones para el lector. ]}

\textcolor{blue}{[Introducir hashes y mencionar los beneficios de utilizarlos para almacenar informacion.]}



Blockchain es una tecnología que proporciona una base de datos que funciona como un registro inmutable, trazable y descentralizado. Puede ser utilizada para abordar la reproducibilidad de ML de la siguiente forma:

\begin{itemize}
    \item \textbf{Registro inmutable:} Permite registrar cada etapa de un pipeline de ML, incluyendo configuraciones de hiperparámetros, preprocesamientos de datos y resultados. \textcolor{blue}{[Ver opción de incluir un hash del entorno de ejecucion para capturar versiones de software.]}
    
    \item \textbf{Transparencia:} El registro queda asentado en la blockchain con su timestamp, permitiendo ser auditable y  verificable, favoreciendo la replicabilidad de un modelo.
    
    \item \textbf{Integridad de los datos:} El almacenamiento de hashes de datos y modelos en la Blockchain permiten validar resultados obtenidos.
\end{itemize}

\section{Información a trazar}
\textcolor{blue}{[Además de almacenar hashes en blockchain, los datos y código deben estar disponibles a terceros.]}

Para asegurar la reproducibilidad de modelos de ML, es fundamental trazar y registrar lo siguiente:

\begin{enumerate}
    \item \textbf{Entorno}
    \begin{itemize}
        \item Detalles del entorno de entrenamiento. Hardware y Software. Librerías y versiones.
    \end{itemize}
    \item \textbf{Datos}
    \begin{itemize}
        \item Hash de datos utilizados para entrenamiento, validación y test.
        \item Preprocesamiento de datos. Detalles de modificaciones y transformaciones al conjunto inicial de datos.
    \end{itemize}
    \item \textbf{Modelo}
    \begin{itemize}
        \item Especificaciones de la arquitectura del modelo.
        \item Hiperparámetros. 
    \end{itemize}
    \item \textbf{Evaluación}
    \begin{itemize}
        \item Metricas de evaluación del modelo. Entrenamiento, validación y test.
    \end{itemize}
    \item \textbf{Documentación}
    \begin{itemize}
        \item Informes y documentación del experimento y resultados.
    \end{itemize}
    
\end{enumerate}


\section*{BlockchainTracer}

\textbf{BlockchainTracer} es un paquete de Python diseñado para trazar información sensible y flujos de procesos en la blockchain.

Aprovecha las propiedades inherentes de la blockchain —inmutabilidad, transparencia, disponibilidad y trazabilidad— para registrar y auditar pasos secuenciales en cualquier proceso. Es ideal para aplicaciones que requieren registros verificables de acciones o rastros de datos sensibles.

\subsection*{¿Qué hace?}

Permite guardar pasos secuenciales de cualquier cosa. Casos de uso:

\begin{itemize}
  \item Mejorar la reproducibilidad de modelos de Machine Learning (principal idea).
  \item Subir hashes de archivos grandes de datos.
  \item Rastrear donaciones de ONGs.
  \item Mejorar la trazabilidad de cadenas de suministro.
  \item Guardar información importante de estudios científicos.
  \item Prueba de autoría (resultados con dirección y timestamp).
  \item Cualquier texto.
  \item Lo que quieras.
\end{itemize}

\subsection*{Implementación}

Una única clase Python multipropósito.

\section*{Roadmap}

\subsection*{Etapa 1}

\begin{enumerate}
  \item \textbf{Leer bibliografía relacionada sobre reproducibilidad en ML.}
  \begin{itemize}
    \item \textit{A Survey of Data Provenance in e-Science}
    \item \textit{Ensuring Trustworthy Neural Network Training via Blockchain}
    \item \textit{Towards Enabling Trusted Artificial Intelligence via Blockchain}
    \item \textit{BlockFlow: Trust in Scientific Provenance Data}
    \item \textit{ProML: A Decentralised Platform for Provenance Management of ML Systems}
    \item \textit{Blockchain Based Provenance Sharing of Scientific Workflows}
    \item \textit{Improving Reproducibility in Machine Learning Research (2021)}
    \item \textit{Reproducibility in Machine Learning-Driven Research (2023)}
    \item \textit{Leakage and the Reproducibility Crisis in ML-based Science (2023)}
    \item \textit{reforms: Reporting Standards for ML-based Science (2023)}
    \item \textit{Traceability for Trustworthy AI: A Review of Models and Tools (2021)}
    \item \textit{Reproducibility in PyTorch}
    \item \textit{Advancing Research Reproducibility in ML through Blockchain Technology (2024)}
    \item \textit{Promoting Distributed Trust in ML and Simulation via Blockchain}
    \item \textit{Blockchain Analytics and Artificial Intelligence}
    \item \textit{Automatically Tracking Metadata and Provenance of ML Experiments}
    \item \textit{Reproducibility in ML-based Research: Overview, Barriers and Drivers (2024)}
    \item \textit{Model Cards for Model Reporting}, \textit{Datasheets, Nutrition Labels, Factsheets}
    \item \textit{ML Reproducibility Tools and Best Practices}
  \end{itemize}

  \item \textbf{Especificar diferenciadores de este trabajo.}
  \begin{itemize}
    \item Trazabilidad de modelos ML en blockchains EVM mediante una API Python.
    \item Código abierto.
    \item Adherencia a estándares de reproducibilidad de estudios previos.
    \item Capacidad de trazar otros procesos, pero con foco en ML.
    \item Trazado del entorno computacional donde se entrenó el modelo.
    \item Uso de IPFS o Arweave para datos grandes (guardar solo el hash en blockchain).
  \end{itemize}

  \item \textbf{Afinar el caso de ML: ¿Qué se necesita para una buena reproducibilidad?}
  \begin{itemize}
    \item Checklist de reproducibilidad de NeurIPS 2019.
    \item Estructura JSON con cada configuración del pipeline ML (hardware, entorno, preprocesado, hiperparámetros, semillas, métricas, versiones de paquetes, etc.).
    \item Info sheet del modelo (según el paper de la crisis de reproducibilidad).
    \item ¿Entorno estandarizado? ¿Docker siempre es necesario?
    \item Checklist de \textit{reforms}.
    \item Perfil de descripción mínima.
    \item Model Cards.
    \item MLFlow para logging de experimentos (permite comparar versiones y funciona con sklearn, XGBoost, etc.).
  \end{itemize}

  \item \textbf{Decidir qué trazar.} ¿Usar todos los estándares y eliminar repeticiones? ¿Elegir uno? ¿Centrarse en narrativas, parámetros u otro subconjunto? Pensar en conjuntos. Model Cards y AutoML ya están testeados, se pueden usar como base.

  \item \textbf{Ofrecer a los usuarios lo necesario para reproducir modelos.}
  \item \textbf{Asegurar facilidad de uso y funcionamiento.}
  \begin{itemize}
    \item Código Python accesible para usuarios técnicos no expertos en blockchain.
    \item Integración con blockchains EVM.
    \item Seguridad del manejo de llaves privadas.
    \item Testing.
  \end{itemize}
\end{enumerate}

\subsection*{Etapa 2}
\begin{enumerate}
  \setcounter{enumi}{6}
  \item Resolver qué hacer con código y binarios.
  \item Integración con IPFS o Arweave para archivos grandes.
\end{enumerate}

\subsection*{Etapa 3}
\begin{enumerate}
  \setcounter{enumi}{8}
  \item Frontend para escalabilidad (uso por personas no técnicas).
  \item Smart contract para descentralizar la ejecución del código.
  \item Extensión a otras blockchains públicas.
  \item Soporte para blockchains privadas.
  \item Opción para trazar datos con una nueva dirección (anonimato).
  \item Soporte para más RPCs además de Infura.
  \item Automatizar el completado de info sheet del modelo. ¿Usar un LLM? ¿Generar .tex?
\end{enumerate}




%\section{Aplicación}
%\textcolor{blue}{[Hay un primer script que implementa un modelo de ML y guarda hashes de cada una de las partes del Pipeline de ML. Está orientado para utilizar una blockchain con EVM. Se comienza con una testnet (Sepolia) para realizar transacciones en una blockchain sin costo.
%Una alternativa para no realizar una transaccion para guardar un hash por cada parte es obtener un único hash que describa a todo el proceso.
%]}

%\textcolor{blue}{[Se puede pensar con otros enfoques para que la implementacion sea más fácil para el usuario, ie: que no tenga que hashear cada etapa del pipeline por su cuenta.]}

\section{comentarios}
tener en cuenta requirements.txt para reproducibilidad 

\section{Apéndice}

\subsection{Plataformas Blockchain}

Evaluamos dos posibilidades: blockchains basadas en EVM y la blockchain Algorand.

\subsubsection{Ethereum Virtual Machine}
\begin{itemize}
    \item Ethereum y Polygon, algunos ejemplos de blockchains con EVM.
    \item Contratos inteligentes escritos en Solidity, el lenguaje más aceptado en blockchain.
\end{itemize}


\subsubsection{Algorand}
\begin{itemize}
    \item Ofrece bajos fees de transacciones y rápida validación de las mismas.
    \item Permite volumen considerable de transacciones.
    \item Los contratos inteligentes se pueden escribir en Python.
\end{itemize}
\end{document}
